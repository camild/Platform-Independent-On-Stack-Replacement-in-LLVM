 % !TEX root = article.tex

\section{Conclusions}
\label{se:conclusions}

In this paper, we have presented an OSR framework that introduces novel ideas and combines features of extant techniques that no previous solution provided simultaneously. Relevant aspects include platform independence~\cite{lameed2013modular}, generation of highly optimized continuation functions~\cite{fink2003design}, and performing deoptimization without the need for an interpreter~\cite{bebenita2010spur}. Two novel features we propose are OSR with compensation code, which allows to extend the range of points where OSR transitions can be fired, and the ability to inject OSR points at arbitrary locations. Using these features, we have shown how to improve the state of the art of \feval\ optimization in MATLAB virtual machines. We have also investigated the feasibility of our approach in LLVM, showing that it is efficient in practice.

%\ifx\noauthorea\undefined
\paragraph{Acknowledgements.}

We wish to thank Jan Vitek, Petr Maj, Karl Millar, and Olivier Fl{\"u}ckiger for many enlightening discussions. We are especially grateful to Jan for sparking our interest in this exciting line of research. % during a pleasant visit at Purdue University and for interesting conversations in many other occasions.
%\fi
