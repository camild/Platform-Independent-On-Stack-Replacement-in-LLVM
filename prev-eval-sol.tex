% !TEX root = article.tex

\subsection{Comparison to previous approaches}
\label{ss:prev-eval-sol}

Lameed and Hendren~\cite{lameed2013feval} proposed two dynamic techniques for optimizing \feval\ instructions in McVM: {\em JIT-based} and {\em OSR-based} specialization, which work as follows.

\paragraph{JIT-based specialization.}  

[...]

\paragraph{OSR-based specialization.} This approach triggers an OSR in a function $f$ when a loop containing an \feval\ becomes hot. At that time, an optimized version $f'$ of $f$ is generated and control is diverted to $f'$. $f'$ is created by an optimizer that attempts to replace an \feval$(g,x,y,z...)$ instruction with a direct call $g(x,y,z,...)$ with unboxed parameters. The optimizer works at IR level and uses the actual values of $g$ and its arguments (known at the OSR time) to specialize the code. The direct call to $g$ is guarded by a condition that checks if $g$ and the type of its parameters remains the same as observed when the OSR was fired. If the guard fails, the code falls back to executing the original \feval\ instruction.

This method overcomes the limitations of JIT-based specialization, supporting optimization of \feval$(g,...)$ calls in functions that do not receive $g$ as a parameter. However, it is substantially slower for two main reasons:

\begin{enumerate}
\item When a function $f$ is first compiled from MATLAB to IR by McVM, the functions it calls via \feval\ are unknown and the type inference engine is unable to infer the types of their returned values. Hence, these values must be kept boxed in heap-allocated objects and handled with slow generic instructions in the IR representation of $f$ (suitable for handling different types). For this reason, the optimized continuation functions $f'$ generated at OSR points in McVM inherit the slow generic instructions of $f$.
\item Guard computation in $f'$ can be rather expensive, as it may require checking many parameters.
\end{enumerate}

\noindent Our approach combines the flexibility of OSR-based specialization with the efficiency of JIT-based specialization, answering an open question raised by Lameed and Hendren~\cite{lameed2013feval}. Similarly to OSR-based specialization, it does not place restrictions on the functions that can be optimized. Furthermore, since the dynamic code optimization works at IIR (rather than IR) level, type inference can replace generic istructions with specialized instructions, removing the main source of inefficiency of Lameed and Hendren's OSR-based specialization.

{\tt [Daniele --> text moved from case-study.tex]} Compared to the OSR-based approach by Lameed and Hendren, our solution is cheaper because the types for the other arguments do not need to be cached or guarded: as we will see later on, the type inference engine will compute the most accurate yet sound type information in the analysis of the optimized IIR where direct calls are used.

\ifdefined\fullver
The first one is based on OSR: using the McOSR library~\cite{lameed2013modular}, \feval\ calls inside loops are instrumented with an OSR point and profiling code to cache the last-known types for the arguments of each \feval\ instruction. When an OSR is fired at run-time, a code generator modifies the original function by inserting a guard to choose between a fast path containing a direct call and a slow path with the original \feval\ call. The second technique is less general and uses value-based JIT compilation: when the first argument of an \feval\ call is an argument of the enclosing function, the compiler replaces each call to this function in all of its callers with a call to a special dispatcher. At run-time, the dispatcher evaluates the value of the argument to use for the \feval\ and executes either a previously compiled cached code or generates and JIT-compiles a version of the function optimized for the current value.

Although the OSR-based approach is more general, it generates much less efficient code compared to the JIT-based one for three reasons:
\begin{enumerate}
\item since the function called through \feval\ is unknown at compile time, the type inference engine is unable to infer types for the returned values, so the compiler has to generate generic instructions (suitable for handling different types) for the remainder of the code;
\item guard computation is expensive, not only because the value of the first argument, but also the types of the remaining arguments have to be checked to choose between the fast and the slow path;
\item since an \feval\ is executed through the interpreter, arguments are boxed to make them more generic before the call.
\end{enumerate}

The first one in particular is a major source of inefficiency for the OSR-based approach, since the benefits from replacing the call to the interpreter's \feval\ dispatcher with a direct call are limited compared to the optimization opportunities deriving from a better type inference on the whole body of the function. In fact, as they operate on boxed values, instructions for generic-type variables are inherently much less efficient than their counterparts for [arrays of] primitive types. While the JIT-based approach is preferable as it generates much better code, on the other hand it cannot be applied to cases in which the first argument $f$ to \feval\ is not passed as argument to the enclosing function $g$. Some possible scenarios for MATLAB programs are:
\begin{itemize}
\item $f$ is an {\tt inline} or an anonymous function defined in $g$;
\item $f$ is the return value from a previous call in $g$ to another function;
%\item $f$ is retrieved from a data structure~\cite{lameed2013feval};
\item $f$ is a constant string containing the name of a user-defined function (a typical misuse of \feval ~\cite{radpour2013refactoring}).
\end{itemize}
 
Lameed and Hendren conclude their paper by stating, ``It would be interesting to look at future work that combine the
strengths of both approaches". In the remaining part of this section, we extend McVM by implementing a novel optimization mechanism for \feval\ based on our OSR technique: we will show that our mechanism is as efficient as their JIT-based approach in terms of quality of generated code, and is even more general than their OSR-based approach, as it can optimize also \feval\ calls not enclosed in a loop.
\fi

