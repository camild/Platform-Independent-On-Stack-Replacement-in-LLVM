% !TEX root = article.tex

On-Stack Replacement (OSR) is a technique for dynamically transferring execution between different versions of a function at run time.
%suspending the execution of a function and continuing with a different version of the code. 
OSR is typically used in virtual machines to interrupt a long-running function and recompile it at a higher optimization level, or to replace it with a different one when a speculative assumption made during its compilation no longer holds.

In this paper we present a framework for OSR that introduces novel ideas and combines features of extant techniques that no previous solution provided simultaneously. New features include OSR with compensation code to adjust the program state during a transition and the ability to fire an OSR from arbitrary locations in the code. Our approach is platform-independent as the OSR machinery is entirely encoded at a compiler's intermediate representation level.

%Compared to extant OSR techniques, our approach is platform-independent as transitions are encoded entirely at the Intermediate Representation (IR) level; it supports OSR-point insertion at arbitrary locations in a function, 

%and it allows a function version reached via OSR to fire an OSR itself, either to a more optimized version or to a less optimized one from which it was derived.

We implement and evaluate our technique in the LLVM compiler infrastructure, which is gaining popularity as Just-In-Time (JIT) compiler in virtual machines for dynamic languages such as Javascript, MATLAB, Python, and Ruby. As a case study of our approach, we show how to improve the state of the art in the optimization of the \feval\ instruction, a performance-critical construct of the MATLAB language.

%We also present a case study on the integration of our technique in the McVM virtual machine for MATLAB, showing significant speedups from the run-time optimization of a typical construct of the language.
  
  
  
  
  
  
  
  
  